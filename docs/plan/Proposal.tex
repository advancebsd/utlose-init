\documentclass{article}

\usepackage[margin=0.75in]{geometry}
\usepackage[style=numeric-comp, sorting=none, backend=biber, doi=false, isbn=false, url=false]{biblatex}
\usepackage{minted}
\setminted{autogobble}
\newcommand*{\bibtitle}{Bibliography}
\newenvironment{manpage}{\ttfamily}{\par}
\usepackage{longtable}

\addbibresource{Proposal.bib}

\title{Init System Proposal\\ \large{For all POSIX complient systems}}
\author{James Hobson and Michael Reim}

\begin{document}
\maketitle

\tableofcontents

\setlength{\parindent}{0em}
\setlength{\parskip}{1em}

\section{Motivations}
Proposing new init systems and service management systems in 2022 is surprisingly still a taboo topic.
This space has seen little innovation since the controversially wide adoption of \texttt{systemd} in
linux and I think, rather understandably, no one wants another\footnote{The systemd case may be what
many people immediately think off as it kind of still rages on. It is however not the first fierce
fight over init systems. In fact it feels a lot like the situation when the BSDs adopted \texttt{rc.d}.
There have been fierce arguments including claims of this being "not the BSD way" and threads to fork
the old init system and continue on with it! For some reason or another, people tend to be extremely
touchy when it comes to their \texttt{init} - and probably rightfully so!} init world war. But the war
never really ended; while \texttt{systemd} gained control of vast amounts of territory\footnote{Some
people claim that this happened not due to technical superiority. There is some truth to that and
especially in the fight over \textit{Debian} corporate actors (\textit{Red Hat} for \texttt{systemd}
and \textit{Canonical} in favor of \texttt{upstart}) were not hard to spot. But it is also true that
\texttt{systemd} \textit{does} provide a lot of value over \texttt{SysV init}.} and a kind of cease
fire was put in place.

Despite having lost all the mainstream Linux distributions to systemd, enough people continue their
resistance to make maintaining several alternative distributions feasible.\footnote{There are those
like \textit{Slackware} which has used its BSD-inspired init system before \texttt{systemd} entered the
stage and continues to do so. Other such examples are \textit{Gentoo} (by default) and \textit{Alpine
Linux} which uses \texttt{openRC} and \textit{Void Linux} that adopted \texttt{runit}. There are
however even new \textit{systemd-free} distributions which consider this an important feature. Notable
examples are the Debian fork \textit{Devuan} and the Arch Linux fork \textit{Artix}.} However there
also exists a minority who still felt oppressed and jumped over to the BSDs to escape the creep of
systemd. But this leaves the BSDs in a situation where, if they wanted to innovate their init system
and service management, they would have to to try and carefully avoid emoting any PTSD in the refugees
they got from Linux. These non-mainstream open source communities are likely too small to further
fragment and remain viable\footnote{\textit{FreeBSD} as the by far largest player has seen several
failed attempts of getting \texttt{openRC} into the base system. \textit{TrueOS}, a friendly fork that
received corporate sponsorship has made completed the switch -- but it died not too long afterwards.
\textit{GhostBSD}, a desktop-focused distribution of FreeBSD picked it up and maintained it for years
but eventually even migrated back mainly because of the maintenance burden.}. Therefore an unspoken
policy of init-system complacency has been adopted (i.e. it's regarded simple, well-known and for a lot
use cases in fact \textit{good enough}).

While some people argue that GNU/Linux is the only remaining Unix-like operating system that still
matters, things are changing. The last couple of years have seen renewed interest in *BSD and an
increase in newcomers. \textit{FreeBSD} is attractive to many due to its superb integration of ZFS, its
proven jails system and many other features. \textit{OpenBSD} is relatively well known for its
security-related innovations. \textit{NetBSD} is a little less visible, but people are also finding
their way to it. And even \textit{Dragonfly BSD} has gone from a relatively unknown operating system to
one that has started to out perform Linux in some benchmarks\footfullcite{dfbsd-phoronix}.

However there are a few things that arguably hinder a renaissance of the BSD. One thing that
\texttt{systemd} has shown is that proper service management is key to speed and effective resource
management. Therefore it may be time to take another look at init systems and service management so
that the BSDs can have an up-to-date answer to the systemd problem and so that systemd has competition
in it's domain. There are a couple of candidates to potentially fill that gap. It makes sense to take a
closer look at them first.

\section{Other Software in the Domain}
\subsection{Primordial Unix init}
\nocite{unix-ii}

This init system, also known simply as \textit{init} was there even in the earliest editions of Research U\textsc{nix} but evolved over time.

\subsubsection{Summary}
The old U\textsc{nix} init is extremely primitive. A benefit to this approach is that it's easy to
understand as well as very stable. But the complete lack of service management has made it obsolete for
a very long time. Some software, such as Apache HTTPd, came with management shell
scripts\footfullcite{app-service} for service management back in the day, but for programs that didn't,
your only option was to send the correct signal to the process via \texttt{kill}. For admins unfamiliar
with this kind of management it's in fact easier to restart the whole machine after some service died
than to figure out how to start it again!

\subsubsection{Details}

In the original First Edition of Research U\textsc{nix}, init was a self-contained program that was
responsible for all the tasks required to bring up the system. Since it is of high historical value, we
replicate the information from the manual here. Keep in mind that the sections weren't what we're used
to today and those sections were denoted in Roman numbers. Two other curious things are that the init
binary used to be located in /etc and that originally there was no \textit{group} bit for file
permissions.

Here is the manual for init as of the First Edition\footnote{Taken from here: \url{https://www.tuhs.org/Archive/Distributions/Research/Dennis_v1/man71.pdf}}:\pagebreak

\begin{manpage}
	
	\begin{longtable}{ll}
		11/3/71 & \hspace{17em}/ETC/INIT (VII)\\\\
		
		NAME & init -- process initialization\\\\
		
		SYNOPSIS & --\\\\
		
		DESCRIPTION & \textit{init} is invoked inside UNIX as the last step in\\
		& the boot procedure. It first carries out several\\
		& housekeeping duties: it must change the modes of\\
		& the tape files and the RK disk file to 17, be-\\
		& cause if the system crashed while a \textit{tap} or \textit{rk}\\
		& command was in progress, these files would be\\
		& inaccessible; it also truncates the file\\
		& /tmp/utmp, which contains a list of U\textsc{nix} users,\\
		& again as a recovery measure in case of a crash.\\
		& Directory \textit{usr} is assigned via \textit{sys} \textit{mount} as\\
		& resident on the RK disk.\\\\
		
		& \textit{init} then forks several times so as to create one\\
		& process for each typewriter channel on which a\\
		& user may log in. Each process changes the mode\\
		& of its typewriter to 15 (read/write owner,\\
		& write-only non-owner; this guards against random\\
		& users stealing input) and the owner to the\\
		& super-user. Then the typewriter is opened for\\
		& reading and writing. Since these opens are for\\
		& the first files open in the process, they receive\\
		& the file descriptors 0 and 1, the standard input\\
		& and output file descriptors. It is likely that\\
		& no one is dialled in when the read open takes\\
		& place; therefore the process waits until someone\\
		& calls. At this point, \textit{init} types its "login:"\\
		& message and reads the response, which is looked\\
		& up in the password file. The password file con-\\
		& tains each user's name, password, numerical user\\
		& ID, default working directory, and default shell.\\
		& If the lookup is successful and the user can sup-\\
		& ply his password, the owner of the typewriter is\\
		& changed to the appropriate user ID. An entry is\\
		& made in /tmp/utmp for this user to maintain an\\
		& up-to-date list of users. Then the user ID of\\
		& the process is changed appropriately, the current\\
		& directpry is set, and the appropriate program to\\
		& be used as the Shell is executed.\\\\
		
		& At some point the process will terminate, either\\
		& because the login was successful but the user has\\
		& now logged out, or because the login was unsuc-\\
		& cessful. The parent routine of all the children\\
		& of \textit{init} has meanwhile been waiting for such an\\
		& event. When return takes place from the \textit{sys}\\
		& \textit{wait}, \textit{init} simply forks again, and the child pro-\\
		& cess again awaits a user.\\\\
		
		& There is a fine point involved in reading the\\
		& login message. U\textsc{nix} is presently set up to han-\\
		& dle automatically two types of terminals: 150\\
		& baud, full duplex terminals with the line-feed\\
		& function (typically, the Model 37 Teletype termi-\\
		& nal), and 300 baud, full duplex terminals with\\
		& only the line-space function (typically the GE\\
		& TermiNet terminal). The latter type identifies\\
		& itself by sending a line-break (long space) sig-\\
		& nal at login time. Therefore, if a null charac-\\
		& ter is received during reading of the login line,\\
		& the typewriter mode is set to accommodate this\\
		& terminal and the "login:" message is typed again\\
		& (because it was garbled the first time).\\\\
		
		& \textit{Init}, upon first entry, checks the switches for\\
		& 73700. If this combination is set, \textit{init} will\\
		& open /dev/tty as standard input and output and\\
		& directly execute /bin/sh. In this manner, U\textsc{nix}\\
		& can be brought up with a minimum of hardware and\\
		& software.\\\\
		
		FILES & /tmp/utmp, /dev/tty0 ... /dev/ttyn\\\\
		
		SEE ALSO & sh\\\\
		
		DIAGNOSTICS & "No directory", "No shell". There are also some\\
		& halts if basic I/O files cannot be found in /dev.\\\\
		
		BUGS & --\\\\
		
		OWNER & ken, dmr
	\end{longtable}

\end{manpage}



\subsection{BSD rc}
BSD innovated by adding \texttt{rc.local}. This separated system init and user specified
init, thus removing the anxiety sysadmins would face each time they needed to update the
system. Apart from that change, it was pretty much the same as old Unix init\footfullcite[Research Unix-style/BSD-style]{wiki-init}
and so inherited the lack of features with respect to service management.


\subsection{SysV init}
SysV init added the concept of runlevels\footfullcite[SysV-Style]{wiki-init} and basic 
service management\footfullcite{man-service}. A run level is a defined state that includes
a set of services which have to be running (or have to not be running). Changing run level
is a simple as stopping the services that require stopping and starting the ones
that are not running. The issue with SysV-style init is it's implementation and
how basic the service management is. It over uses symlinks in the file system to determine order,
it uses PID files to keep track of processes. On top of this, the service management is
quite primitive.

\subsection{BSD rc.d}
It provides the main benefit of breaking up formerly monolithic rc into several
init scripts. Unlike SysV init, it is configured centrally in a single (from the user
perspective) file: rc.conf. It also features the \texttt{rcorder(8)} tool and can generally
figure out itself what the right order to start services in is. Thus it avoids the most
serious problems with SysV init. But it's somewhat limited;
there is no service supervision, service status is not actually reliable and it's not helpful
to get a full system overview. Parallelism is an afterthought and a somewhat recent feature
addition.

\subsection{OpenRC}
Good: Introduces some convenience functions like rc-status. Was conceived with allowing
for parallelism. It can work together with other programs to allow for service supervision.
Bad: Picked up the SysV folly of service enabling via symlinks. It is still unreliable with
service status. By itself it's not able to supervise services.

\subsection{Upstart}
Good: Compatibility with SysV init scripts. Asynchronous, event-driven nature. Service
supervision. Bad: Hm! It's been a while\dots I don't remember off the top of my head what I
disliked. Would need to look into it again.

\subsection{Systemd}
Good: Transition from init scripts to unit files. Service supervision. Bad: Braindead
feature creep (for a PID 1 process!!). Very strange random defects (= unreliable). There's
much more both for good and bad, but these are the main 4 points about systemd for me.

\section{Proposal for Utløse}
The aims of this project is to write a research init system called Utløse. It will provide
two things: A way to start services and a way to monitor and manage them.
These components have the following aims
\begin{itemize}
    \item To abstract away order and state from the system
    \item To start the minimum possible number of services
    \item To be as parallel as possible
    \item To assist the user at every complexity level
    \item To allow for maximum customisability
    \item To be fail-safe. If PID 1 crashes, \textit{hopefully}, it will not require a reboot.
    \item To be POSIX compliant and as flexible with licensing as to not lock ourselves to a particular operating system
    \item To intergrate well into package managers
    \item To allow for central monitoring and control for distributed systems
\end{itemize}

Unix has a long history of using programming language theory to solve problems. As soon as
lex and yacc, were available, tools such as awk were written. In the case of awk, a language
for text processing and report generation was written. The language abstracts away all that
gets in the way of it's aims so that it can do what it does well. This is in contrast to most
of the available init systems, which tend to be build upon pre-existing technologies (such
as shell and the file system) or use extremely generic configuration formats (systemd uses
the ini format). As a result, very little complexity is abstracted away. We propose a
new language designed for service management.

The first idea for this language is that dependencies are captured in the syntax. The next idea
is that the language is lazily evaluated. This means that everything is started as late as possible
(if started at all). Automatic parallelism and lazy service starting should mean lighting fast
boot times and minimum resource usage. The language will be garbage collected. But what is
the garbage? The garbage is services that are currently not in use. Will maintain minimum
resource usage. Another idea is finite resource limits. Consider a large number of services
that heavily use the disk or network during initialisation. We could tell the scheduler that
these services take from a finite resource. The scheduler will make sure that only a set number
of these services run at once. This should mean that hardware bottlenecks in the system do not
end up slowing down boot.

The language will also allow for other interpreters to be used in the definition of
services. If you have a bunch of complex procedures that need to be run to calculate how a service
must start, then you can use haskell, lua python (etc). Or you can keep it simple and let it
default to shell.

We may explore disowning children processes. This has some limitations, but possibly would allow
the daemon to fail without the system being rendered unusable.

\subsection{Architecture}
There will be two executables:
\begin{itemize}
  \item \texttt{utlosed}
  \item \texttt{utloseadm}
\end{itemize}
Where \texttt{utlosed} will be the daemon that controls the starting and maintaining of services, and
\texttt{utloseadm} which is the frontend to be used by both users and other applications.
The two applications will be connected via ZeroMQ\footfullcite{zmq-home} sockets. There will be request and reply sockets
for control, but also publisher and subscribe sockets. These will be used for distributed logging and alarms.
ZeroMQ is a good choice because of curveZMQ\footfullcite{cmq-home}. Encrypted sockets (to allow for distributed control)
can be achieved with only minor changes to the code base.

The messaging protocol is not decided yet, but it will be open. This will allow for third parties to easily
write other frontends, or log collectors.

\subsubsection{Logging and alarms}
All output of the services will be redirected back to \texttt{utlosed} which will both write to a rotating log
(configurable) and broadcast. The channel will be configurable, but the default will be: \texttt{logs/hostname/service}.
This means that any logging client can filter for logs, logs from a specific host, and logs from a specific service on a specific
host.

Logs written to \texttt{stderr} (and crash reports detected by \texttt{utlosed}) will also be reported on \texttt{alarms/hostname/service}
allowing for a service to listen specifically for things that are going wrong.

All logs currently held on disk may be requested via the request response socket interface. This allows remote clients to both
respond to live events happening (via subscription), without having to be running 100\% of the time to get a full history.

\subsubsection{Configuration}
Configuration is split into three parts:
\begin{enumerate}
  \item Configuration of \texttt{utlosed}
  \item Configuration of \texttt{utloseadm}
  \item Configuration of the services.
\end{enumerate}

Configuration of the first two will be found in \texttt{/etc/utlose}, but service configuration can be split
accross locations. This is to appease some package managers, which like to place everything in \texttt{/usr}

By default, \texttt{utloseadm} will assume that it is monitoring a local instance of utlose. This can be changed
by adding hosts to \texttt{/etc/utlose/utloseadm.toml}.

The config for \texttt{utlosed} will be stored in \texttt{/etc/utlose/utlosed.toml}, and will be used to turn on or off features,
change time outs and specify where the service files are located. This will also be where the default \textit{rule} is specified.

There are a few approaches to service management config that could be appropriate. I imagine that a few will be supported and
the one used will be configured in \texttt{/etc/utlose/utlosed.toml}

\begin{itemize}
  \item{Subvolume based upgrades: File Systems such as ZFS and BTRFS support snapshots of subvolumes. In this option, a snapshot is created
    every time utløse successfully loads a new complete service configuration. If it fails, it goes through the snapshots until it finds one that
    works.}
  \item{Different locations: In this option, the package manager submits changes to Utløse which are accepted or rejected. The mechanism here would be
    that the package manager puts a service file in some pre-agreed location and calls a subcommand of \texttt{utloseadm} to query the new service.
    The query can be simple, such as ``Is this well formed?'' But more complicated queries can also be supported. If all the queries pass, the new service
    files are included. If not, they are rejected and the package manager warned.}
  \item{Nothing special: Keep it simple stupid! One location for service files.}
\end{itemize}

\subsection{Interconnecting instances}
As outlined above, logs can be broadcast to other instances, but this is not where the interconnecting ends. Utløse instances
can be chained and control commands can be send down the chain. The best way to explain this is by example:

Consider a container (such as a jail), on a virtual machine in a datacenter. On this container, we run NGINX.
The Utløse instance on the container has a service named NGINX. But we have connected this instance with the Utløse instance
in the VM.

The VM starts this container via Utløse and calls the container ``frontend''. But because the instances are connected,
on the VM, we can also see the service \texttt{frontend/nginx}. If allowed, from the VM, we can issue commands to the \texttt{utlosed}
daemon using the \texttt{utloseadm} on the vm, such as \texttt{utloseadm status frontend/nginx}.

The VM is on a node in a datacenter. It's name is ``FreeBSD\_12''. The node's instance is connected to the instance on the VM and so
on the VM we can see the NGINX service as \texttt{FreeBSD\_12/frontend/nginx}. We also have a computer in charge of collecting logs.
This computer can query the node in much the same way.

Of course, permission and security are of paramount importance. For this reason, the connection between the nodes are end to end
encrypted and what a node up the hierarchy can do is controlled by the \texttt{utlosed.toml} file.

\subsection{Configuration Language}
The configuration language is inspired by the shake build system EDSL. Services are defined within targets, both of which
can be dependencies of services. Target definitions can be split allowing for configuration to be split accross files and
are defined with capital letters at the begining of their names. Services are defined within targets and cannot be split.
The general syntax is:

\begin{minted}{haskell}
  Target => -- Name of target
    wants [service1, service3] -- what needs starting in target.

    service1 => -- Service name
      needs [someTarget, someService] -- dependencies
      -- Service definition follows

    service2 =>
      ...

    ...
\end{minted}
both \texttt{wants} and \texttt{needs} are optional. But you will probably want to use \texttt{needs}
unless you are writing a service to be started right after utløse starts!

So how can a package possibly add a service? It seems like it requires to \textit{edit} a file
instead of adding a standalone one! Do not fear. Targets can be merged accross files for example:

\begin{minted}{haskell}
-- /etc/utlose/services.utl.d/10_bla.utl
  Target =>
    wants [foo]

    foo => 
      ...

-- /etc/utlose/services.utl.d/20_other.utl
  Target =>
    wants [bar]

    bar => 
      ...

\end{minted}
Is equivalent to
\begin{minted}{haskell}
  Target =>
    wants [foo, bar]

    foo => ...
    bar => ...
\end{minted}
Files are included in lexicographical order, so if there is a conflict, the second file takes precedence.
\texttt{/etc/utlose/services.utl} is loaded after all of the ones in the \texttt{services.utl.d} folder
and so takes precedence. This is for user defined services.

\subsubsection{Service settings}
The following things are configurable by a service:
\begin{itemize}
  \item Run user, the user that the service runs as
  \item Service type, default is simple, other option is scripted
  \item Command, what to run!
  \item{
      Socket, this is for the lazy starting of services network serivices. A port is defined and a
      unix domain socket as the proxy}
  \item Sockets, same as socket but accepts a list of socket mappings as opposed to 1.
  \item gcTime, if the service is started lazily, kill it if no activity on the socket after the given time
\end{itemize}

\begin{minted}{haskell}
  SomeTarget =>
    someService =>
      needs [Network]
      user    = "root"
      type    = SIMPLE
      socket  = 0.0.0.0:8080 -> /root/fun.sock
      gcTime  = 1 day
      command = /usr/bin/funservice --listen-on /root/fun.sock
\end{minted}

\subsubsection{Starting a service}
A simple service launches a program and seems the job done. Often, you might want to
wait for something to happen before you deem the job done. How this information is passed to
utløse is up to the programmer, as this is done through scripts:

\begin{minted}{haskell}
  SomeTarget =>
    someService =>
      needs [Network]
      user    = "root"
      type    = SCRIPTED
      socket  = 0.0.0.0:8080 -> /root/fun.sock
      gcTime  = 1 day
      command = 
       [/bin/bash |
          /usr/bin/funservice --listen-on /root/fun.sock &
          while [ ! -f /etc/fun/started ]; do
            sleep 1
          done

          fg
       |]
\end{minted}

You can use any interpreter is appropriate. This should mean that no program needs to depend
upon utløse to be integrated well. An aim of this project is portability and this means
software written for an Utløse system is in no way bound to it.

\subsubsection{Throttles}
We may want to limit the number of web services that start simultaneously on boot.
Maybe this is because we have limited bandwidth and it would slow down boot.
We might also have disk heavy services that we want to limit.
For this we use finite resources:

\begin{minted}{haskell}
net  <- resource 10 -- 10 network services max at a time
disk <- resource 4

Product =>
  frontend net =>
    ...
  database net disk =>
    ...
\end{minted}

\subsection{Service Management Tool}


\setlength{\baselineskip}{0pt} % JEM: Single-space References

{\renewcommand*\MakeUppercase[1]{#1}%
\printbibliography[heading=bibintoc,title={\bibtitle}]}

\end{document}
